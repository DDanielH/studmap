\chapter{Client}
Der Client ist eine Android-Applikation f�r den Endbenutzer. Es handelt sich hierbei um einen Navigator. 
Wie jedes handels�bliche Navigationsger�t beinhaltet auch unsere Applikation eine Karte, in unserem Fall ein Geb�udeplan, 
da wir uns in unserem Projekt mit Indoor-Navigation besch�ftigen. Dar�ber hinaus kann man einen Zielpunkt w�hlen
und mit Hilfe eines QR-Codes, NFC-Tags oder des Wlans seine Position bestimmen. Sind Start und Zielpunkt bekannt, berechnet der Server
eine Route und teilt diese dem Client mit, welcher diese daraufhin graphisch auf der Karte zur Anzeige bringt. \\
Des Weiteren gibt es Suchfunktionen und Auflistungen von besonders interessanten Orten (Points of Interest).
Zur Benutzung der Wlan-Positionserkennung ist eine Registrierung und Anmeldung an unserem Server von N�ten.

\section{Allgemeine Struktur}
Die Navigator-Applikation wurde wie auch der Collector in Android umgesetzt. Es wird mindestens die Version 4.1 vorausgesetzt.
Es galt folgende Hauptaufgaben zu erf�llen:

\subsection{Positionserkennung}
Wie bereits im Kapitel \nameref{Collector} beschrieben, haben wir uns f�r Positionserkennung anhand von QR-Codes, NFC-Tags und Wlan-Fingerprinting entschieden.
F�r die Positionserkennung mittels NFC und Wlan greifen wir auf unsere Android-Bibliothek der Kernfunktionalit�ten zur�ck. W�hrend f�r die Interpretation des QR-Codes
eine Fremd-Applikation zum Einsatz kommt. Diese muss bei der ersten Benutzung gegebenenfalls installiert werden. \\
Die letztendlich ermittelten Daten interpretieren wir mit Hilfe des Webservices, der wiederum Bestandteil unserer Kernfunktionalit�ten ist.

\subsection{Anzeige und Navigation auf einer Karte}
Wir haben uns f�r die Benutzung einer freien Bibliothek f�r Karten entschieden. Nach intensiver Recherche viel unsere Wahl dabei auf die Javascript Bibliothek d3 von \todo {Thomas: Infos zur Bibliothek}.
Das hat zur Folge, dass unsere Karte lediglich in einer Website platziert ist, die mittels eine WebView zur Anzeige gebracht wird. Dies hat f�r uns und den Endbenutzer
den gro�en Vorteil, dass �nderung an der Karte nicht ein Update der Applikation nach sich zieht.
Mittels einer definierten Schnittstelle l�sst sich das Javascript aus unserer Android-Applikation heraus bedienen, so dass beispielsweise Positionsermittlungen automatisch 
auf der Karte nachgehalten werden. Dadurch ist die Navigation bei gew�hlten Zielpunkt vollst�ndig.

\subsection{Visuell ansprechende und intuitiv bedienbare Applikation}
Eine Applikation sollte heute intuitiv bedienbar, �bersichtlich und ansprechend sein, damit sie gerne und viel benutzt wird. Wir bauen dabei auf moderne M�glichkeiten des
Android Betriebssystems. Im folgenden Kapitel werden wir diese n�her erl�utern.

\section{Benutzeroberfl�che}
Neben der bereits erw�hnte WebView mit eine h�bschen Karte auf Basis der d3 Bibliothek, ist unsere gesamte Applikation in dunklen T�nen gehalten und bietet klare Linien bei
einer Highlight-Farbe die dem Gr�n der Westf�lischen Hochschule sehr nahe kommt. 
Im Vordergrund der Applikation steht selbstverst�ndlich die Karte, w�hrend grundlegende Funktionen wie die Suche oder der Aufruf des QR-Code-Scanners in der Actionbar geboten werden.
Weitergreifende Funktionalit�ten wie das Wechseln der Ebene oder Anmelden befinden sich in einem Drawer auf der linken Seite der Applikation, der auf Wunsch in das Bild hinein gezogen
werden kann. \\
Zus�tzliche Fenster wie z.B. die Auflistung der Points of Interest werden grunds�tzlich in Dialogfenstern zur Anzeige gebracht, was der �bersichtlich und Navigation durch die Applikation zu Gute kommt. Die Suche ist ein besonderes Event, welches in der Actionbar ausgef�hrt wird und dort die Anzeige ver�ndert, sodass grunds�tzlich Ordnung in der Applikation herrscht.
Neben dem intuitiv Design der Applikation ist selbstverst�ndlich auch die Bedienung der Karte �u�erst benutzerfreundlich. Neben den bekannten M�glichkeiten des Multitouch, beispielsweise zum Zoomen, ist auch die Steuerung der Navigation selbsterkl�rend. Einen gew�nschten Punkt angeklickt, schon kann es los gehen. Optisch ansprechend wird eine Route eingezeichnet und mit der Zielflagge gekennzeichnet.


% TODOS:
% - Funktionsweise
% - UI Aspekte
% - Android Zeug...
% - Verweis auf Anhang f�r Bedienungsanleitung / Installationsanleitung