\chapter{Client}
Der Client ist eine Android-Applikation f�r den Endbenutzer. Es handelt sich hierbei um einen Navigator. 
Wie jedes handels�bliche Navigationsger�t beinhaltet auch unsere Applikation eine Karte, in unserem Fall ein Geb�udeplan, 
da wir uns in unserem Projekt mit Indoor-Navigation besch�ftigen.

Dar�ber hinaus kann man einen Zielpunkt w�hlen
und mit Hilfe eines QR-Codes, NFC-Tags oder des WLANs (experimentell) seine Position bestimmen. Sind Start und Zielpunkt bekannt, berechnet der Server
eine Route und teilt diese dem Client mit, welcher diese daraufhin graphisch auf der Karte zur Anzeige bringt.

Des Weiteren gibt es Suchfunktionen und Auflistungen von besonders interessanten Orten (Points of Interest).
Zur Benutzung der WLAN-Positionserkennung ist eine Registrierung und Anmeldung an unserem Server von N�ten.

\section{Allgemeine Struktur}
Die Navigator-Applikation wurde wie auch der Collector in Android umgesetzt. Es wird mindestens die Version 4.1 vorausgesetzt.
Es galt folgende Hauptaufgaben zu erf�llen.

\subsection{Positionserkennung}
Wie bereits im Kapitel \nameref{cha:Collector} beschrieben, haben wir uns f�r 
Positionserkennung anhand von QR-Codes, NFC-Tags und WLAN-Fingerprinting 
entschieden.
F�r die Positionserkennung mittels NFC und WLAN greifen wir auf unsere 
Android-Bibliothek der Kernfunktionalit�ten zur�ck. W�hrend f�r die 
Interpretation des QR-Codes
eine Fremd-Applikation zum Einsatz kommt. Diese muss bei der erstmaligen Benutzung gegebenenfalls installiert werden.

Die letztendlich ermittelten Daten interpretieren wir mit Hilfe des Webservices, der wiederum Bestandteil unserer Kernfunktionalit�ten ist.

\subsection{Anzeige und Navigation auf einer Karte}
Wie im Kapitel \nameref{cha:MapWebview} beschrieben, haben wir uns f�r die Javascript-Bibliothek \textit{D3} mit dem Plugin \textit{Floor Plan} entschieden, welches die Daten in HTML-Dokumenten visualisiert.

Das hat zur Folge, dass unsere Karte lediglich in einer Website platziert ist, die mittels eine WebView zur Anzeige gebracht wird. Dies hat f�r uns und den Endbenutzer
den gro�en Vorteil, dass �nderungen an der Karte kein Update der Applikation nach sich ziehen.
Mittels einer definierten Schnittstelle l�sst sich das 
\nameref{cha:jsSchnittstelle} aus unserer Android-Applikation heraus bedienen, so dass beispielsweise Positionsermittlungen automatisch 
auf der Karte nachgehalten werden. Bei gew�hltem Zielpunkt beginnt die Navigation automatisch.

\subsection{Visuell ansprechende und intuitiv bedienbare Applikation}
Eine Applikation sollte heute intuitiv bedienbar, �bersichtlich und ansprechend sein, damit sie gerne und viel benutzt wird. Wir bauen dabei auf moderne M�glichkeiten des
Android Betriebssystems. Im folgenden Kapitel werden wir diese n�her erl�utern.

\section{Benutzeroberfl�che}
Neben der bereits erw�hnten WebView mit einer h�bschen Karte auf Basis der \textit{D3} 
Bibliothek, ist unsere gesamte Applikation in dunklen T�nen gehalten und 
bietet klare Linien bei
einer Highlight-Farbe die dem Gr�n der Westf�lischen Hochschule sehr nahe kommt. 
Im Vordergrund der Applikation steht selbstverst�ndlich die Karte, w�hrend grundlegende Funktionen wie die Suche oder der Aufruf des QR-Code-Scanners in der Actionbar geboten werden.
Weitergreifende Funktionalit�ten wie das Wechseln der Ebene oder Anmelden befinden sich in einem Drawer auf der linken Seite der Applikation, der auf Wunsch in das Bild hinein gezogen
werden kann.

Zus�tzliche Fenster wie z.B. die Auflistung der Points of Interest werden 
grunds�tzlich in Dialogfenstern zur Anzeige gebracht, was der 
�bersichtlichkeit und Navigation durch die Applikation zu Gute kommt. Die 
Suche ist ein besonderes Event, welches in der Actionbar ausgef�hrt wird und 
dort die Anzeige ver�ndert, sodass grunds�tzlich Ordnung in der Applikation 
herrscht.
Neben dem intuitiven Design der Applikation ist selbstverst�ndlich auch die Bedienung der Karte �u�erst benutzerfreundlich. Dar�ber hinaus werden die bekannten M�glichkeiten des Multitouch, beispielsweise die Pitch- und Zoom-Geste, unterst�tzt. Ebenso intuitiv bedienbar ist die Steuerung der Navigation. Einen gew�nschten Punkt angeklickt, schon kann es los gehen. Optisch ansprechend wird eine Route eingezeichnet und mit der Zielflagge gekennzeichnet.
