\chapter{Client}
Der Client ist eine Android-Applikation für den Endbenutzer. Es handelt sich hierbei um einen Navigator. 
Wie jedes handelsübliche Navigationsgerät beinhaltet auch unsere Applikation eine Karte, in unserem Fall ein Gebäudeplan, 
da wir uns in unserem Projekt mit Indoor-Navigation beschäftigen. Darüber hinaus kann man einen Zielpunkt wählen
und mit Hilfe eines QR-Codes, NFC-Tags oder des Wlans seine Position bestimmen. Sind Start und Zielpunkt bekannt, berechnet der Server
eine Route und teilt diese dem Client mit, welcher diese daraufhin graphisch auf der Karte zur Anzeige bringt. \\
Des Weiteren gibt es Suchfunktionen und Auflistungen von besonders interessanten Orten (Points of Interest).
Zur Benutzung der Wlan-Positionserkennung ist eine Registrierung und Anmeldung an unserem Server von Nöten.

\section{Allgemeine Struktur}
Die Navigator-Applikation wurde wie auch der Collector in Android umgesetzt. Es wird mindestens die Version 4.1 vorausgesetzt.
Es galt folgende Hauptaufgaben zu erfüllen:

\subsection{Positionserkennung}
Wie bereits im Kapitel \nameref{Collector} beschrieben, haben wir uns für Positionserkennung anhand von QR-Codes, NFC-Tags und Wlan-Fingerprinting entschieden.
Für die Positionserkennung mittels NFC und Wlan greifen wir auf unsere Android-Bibliothek der Kernfunktionalitäten zurück. Während für die Interpretation des QR-Codes
eine Fremd-Applikation zum Einsatz kommt. Diese muss bei der ersten Benutzung gegebenenfalls installiert werden. \\
Die letztendlich ermittelten Daten interpretieren wir mit Hilfe des Webservices, der wiederum Bestandteil unserer Kernfunktionalitäten ist.

\subsection{Anzeige und Navigation auf einer Karte}
Wir haben uns für die Benutzung einer freien Bibliothek für Karten entschieden. Nach intensiver Recherche viel unsere Wahl dabei auf die Javascript Bibliothek d3 von \todo {Infos zur Bibliothek}.
Das hat zur Folge, dass unsere Karte lediglich in einer Website platziert ist, die mittels eine WebView zur Anzeige gebracht wird. Dies hat für uns und den Endbenutzer
den großen Vorteil, dass Änderung an der Karte nicht ein Update der Applikation nach sich zieht.
Mittels einer definierten Schnittstelle lässt sich das Javascript aus unserer Android-Applikation heraus bedienen, so dass beispielsweise Positionsermittlungen automatisch 
auf der Karte nachgehalten werden. Dadurch ist die Navigation bei gewählten Zielpunkt vollständig.

\subsection{Visuell ansprechende und intuitiv bedienbare Applikation}
Eine Applikation sollte heute intuitiv bedienbar, übersichtlich und ansprechend sein, damit sie gerne und viel benutzt wird. Wir bauen dabei auf moderne Möglichkeiten des
Android Betriebssystems. Im folgenden Kapitel werden wir diese näher erläutern.

\section{Benutzeroberfläche}
Neben der bereits erwähnte WebView mit eine hübschen Karte auf Basis der d3 Bibliothek, ist unsere gesamte Applikation in dunklen Tönen gehalten und bietet klare Linien bei
einer Highlight-Farbe die dem Grün der Westfälischen Hochschule sehr nahe kommt. 
Im Vordergrund der Applikation steht selbstverständlich die Karte, während grundlegende Funktionen wie die Suche oder der Aufruf des QR-Code-Scanners in der Actionbar geboten werden.
Weitergreifende Funktionalitäten wie das Wechseln der Ebene oder Anmelden befinden sich in einem Drawer auf der linken Seite der Applikation, der auf Wunsch in das Bild hinein gezogen
werden kann. \\
Zusätzliche Fenster wie z.B. die Auflistung der Points of Interest werden grundsätzlich in Dialogfenstern zur Anzeige gebracht, was der Übersichtlich und Navigation durch die Applikation zu Gute kommt. Die Suche ist ein besonderes Event, welches in der Actionbar ausgeführt wird und dort die Anzeige verändert, sodass grundsätzlich Ordnung in der Applikation herrscht.
Neben dem intuitiv Design der Applikation ist selbstverständlich auch die Bedienung der Karte äußerst benutzerfreundlich. Neben den bekannten Möglichkeiten des Multitouch, beispielsweise zum Zoomen, ist auch die Steuerung der Navigation selbsterklärend. Einen gewünschten Punkt angeklickt, schon kann es los gehen. Optisch ansprechend wird eine Route eingezeichnet und mit der Zielflagge gekennzeichnet.


% TODOS:
% - Funktionsweise
% - UI Aspekte
% - Android Zeug...
% - Verweis auf Anhang für Bedienungsanleitung / Installationsanleitung