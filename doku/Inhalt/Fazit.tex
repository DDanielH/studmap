\chapter{Fazit}

%\section{Lokalisierung innerhalb von Geb�uden}
Im Verlauf unseres Projektes ist deutlich geworden, dass die Navigation und die Lokalisierung innerhalb von Geb�uden ein schwieriges Thema ist. Daher w�hlten wir f�r das Problem der Lokalisierung gleich mehrere Ans�tze und mussten erkennen, dass sowohl die Positionierung mittels Texterkennung der Raumschilder, als auch die Positionsermittlung �ber WLAN-Fingerprints f�r unsere Anforderungen nicht bzw. nicht ausreichend gut funktioniert haben.
Zufriedenstellende Ergebnisse lieferte hingegen die Positionsermittlung
mittels Scannen von QR-Codes und NFC-Tags, die zuvor im Geb�ude an allen
relevanten Orten angebracht werden mussten.

%\section{Verwendete Technologien}
Bei der Umsetzung dieser Ideen haben wir uns f�r ein Backend basierend auf Microsoft Technologien entschieden. F�r die Entwicklung der Datenbankstruktur verwendeten wir das Entity Framework, mit dem die Datenbank automatisch aus unserem Datenmodell generiert werden konnte. Zus�tzlich konnten komplexe Datenbankabfragen einfach umgesetzt und ausgewertet werden.

Des Weiteren haben wir uns bei der administrativen Oberfl�che f�r eine ASP.NET Webapplikation entschieden, in der wesentliche Bestandteile der Benutzeroberfl�che und eine Benutzerverwaltung bereits integriert waren. 
Die Auswahl der Frameworks er�ffnete uns die M�glichkeit, uns von
Anfang an auf die Kernidee des Projekts zu fokussieren.

%\subsection{Zusammenarbeit mit Microsoft}
Begeistert von diesen vielen M�glichkeiten, entschieden wir uns unsere Anwendung in der Windows Azure Cloud zu hosten und meldeten daher einen entsprechenden Studenten Account bei Microsoft an. Leider erhielten wir �ber Wochen kein eindeutiges Feedback von Microsoft, weshalb wir uns letztendlich f�r einen eigenen Windows Server innerhalb der Hochschule entschieden haben.

%Im Frontend haben wir uns f�r eine Android Anwendung entschieden, bei der wir die wesentlichen Bestandteile in Form von Webseiten abgebildet haben. Das f�hrte %zu einer hohen Wiederverwendbarkeit, da die Anzeige der Karte in allen Anwendung benutzt wurde.

%\section{Projektmanagement}
Abschlie�end blicken wir auf ein komplexes Projekt mit vielen Herausforderungen zur�ck. Durch die unterschiedlichen Technologien haben wir alle etwas Neues kennengelernt und weitere wichtige Erfahrung sammeln k�nnen. Innerhalb des Projekts gab es allerdings nicht nur technische Herausforderungen, auch die Projektorganisation selbst, sowie die Zusammenarbeit im Team ist in jedem Projekt eine pr�gende Erfahrung. Gemeinsam haben wir es trotz der komplexen Aufgabenstellung geschafft unser Projektziel zu erreichen. So sind insgesamt drei Anwendungen zur Navigation innerhalb unserer Hochschule entstanden, die mit entsprechenden administrativen Aufwand auch wirklich eingesetzt werden k�nnen.