\section{Positionsermittlung}
In diesem Kapitel wird beschrieben, wie die Position eines Anwenders
auf der Karte bestimmt wird. Dazu werden die im Kapitel
\nameref{cha:Recherche} beschriebenen Verfahren verwendet.

\subsection{QR-Tags}
\label{QR-Tags}
An den R�umen werden QR-Tags angebracht, die eine Zuordnung 
zu einem Knoten auf der erm�glicht. Hier ist zu beachten,
dass die Informationen auf dem QR-Tag auch f�r andere Anwendungen
n�tzlich sein sollen. Deshalb kann hier nicht nur eine Knoten-ID
hinterlegt werden.

Folgendes JSON-Format ist ein m�glicher Kandidat:
\begin{lstlisting}
{
  "General": {
    "Label": "A2.1.10",
    "Name": "Aquarium"
  },
  "StudMap": {
    "NodeId": "12",
    "Url": "https://code.google.com/p/studmap/"
  }
}
\end{lstlisting}

\subsection{NFC-Tags}
\label{NFC-Tags}
Neben den QR-Tags werden auch NFC-Tags an den R�umen angebracht.
Auf diese werden aktuell keine Informationen geschrieben. Die
Knoten werden �ber die NFC-ID des Chips zugeordnet.

Um Benutzer, die die StudMap-App nicht installiert haben zu
informieren wird eine URL auf den NFC-Tags gespeichert. Diese
leitet auf eine Seite mit Informationen �ber den gescannten Raum
und einen Link auf unsere Projektseite. In Zukunft kann hier auch
ein Link auf den Google Play Store hinterlegt werden, um eine
einfache Installation zu erm�glichen.

Die URL sieht z.B. so aus (f�r das Aquarium):
\begin{lstlisting}
http://193.175.199.115/StudMapAdmin/Admin/NodeInfo?nodeId=847
\end{lstlisting}