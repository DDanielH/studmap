\section{QR-Codes}
QR-Codes sind weit verbreitet, einfach zu erstellen und mit vielen Ger�ten 
einzulesen. Dadurch bieten QR-Codes die M�glichkeit Informationen einfach an 
Orten anzubringen und von Maschinen einzulesen.\\
Wir haben uns zwei Konzepte �berlegt, QR-Codes in unserem Projekt einzubauen. 
Dazu haben wir einmal die M�glichkeit betrachtet die Auswertung des QR-Codes 
am Server zu realisieren und mit der M�glichkeit verglichen die Auswertung 
direkt am Client des Nutzers zu implementieren.

\subsection{Serverseitig}
F�r die serverseitige Umsetzung hat gesprochen, dass das Smartphone keine 
Rechenleistung ben�tigt um die QR-Codes zu dekodieren. Des Weiteren wird durch 
eine serverseitige Implementierung vermieden, dass der Nutzer weitere Apps auf 
seinem Smartphone installieren muss.\\
F�r die Implementierung in den Webservice haben wir uns f�r die offene 
Bibliothek \href{http://zxingnet.codeplex.com/}{ZXing.NET} benutzt. Allerdings 
ist uns bei der Implementierung und Testen der Bibliothek direkt aufgefallen, 
dass die Bilder zuvor am Smartphone verkleinert werden m�ssen um Bandbreite zu 
sparen und die Laufzeit der Bibliothek zu verringern. Dadurch wird, obwohl es 
ein Ziel der serverseitigen Umsetzung war, Rechenleistung ben�tigt. Dar�ber 
hinaus mussten wir feststellen, dass die Bibliothek keine zuverl�ssige 
Dekodierung der QR-Codes bietet.

\subsection{Clientseitig}
Im Gegensatz zur serverseitigen Umsetzung wird bei dieser Implementierung die 
gesamte Dekodierung am Client vorgenommen. Dadurch wird die Netzlast 
verringert, der Aufwand am Client aber erh�ht.\\
Hier bot sich zum einen an eine Bibliothek in den Client aufzunehmen, oder 
eine externe Anwendung zum Dekodieren der QR-Codes zu benutzen. Wir haben uns 
schlussendlich dazu entschieden \todo{Entscheidung beim QR-Code Reader 
aufschreiben und warum}