\subsection{NFC-Tags}
NFC ist eine Technologie, auf die wir im Alltag immer h�ufiger sto�en, 
sei es bei Werbung, Bezahl- oder Ticketsystemen.
NFC steht f�r Near Field Communication und besteht aus zwei Komponenten, 
der passiven, dem NFC-Tag, und der aktiven Komponente, beispielsweise dem Smartphone.\\
Die als Datenspeicher fungierenden NFC-Tags werden immer preiswerter und sind zudem relativ
klein, was eine Anbringung an den gew�nschten Orten problemlos erm�glicht. Auf der anderen,
der aktiven Seite stehen immer mehr Smartphones bereit, die diese Technologie unterst�tzen.\\
Die steigende Beliebtheit der NFC-Technologie verdankt diese dem Komfort. Wie der Name bereits
aussagt, reicht schon die N�he der aktiven Komponente zur passiven um Daten zu kommunizieren.
Diesen Komfort bieten wir dem Benutzer, um seine Position dem Navigator mitzuteilen.\\
Des Weiteren lassen sich auf dem NFC-Tag zus�tzliche Informationen hinterlegen, 
die Unwissende auf die Navigationsm�glichkeit aufmerksam machen.

\subsubsection{Umsetzung}
Der gesamte Prozess der Positionsermittlung findet clientseitig statt. 
Zum einen ist das Client-Ger�t unumg�nglich f�r die Kommunikation mit dem NFC-Tag und zum anderen
sind die Datenmengen und der Aufwand der Interpretation sehr gering. \\
Die Android NFC API erm�glicht uns die native Umsetzung der Positionsermittlung f�r Android-Ger�te.
