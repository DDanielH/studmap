\subsection{WLAN Fingerprinting}
Bei der Positionierung des Nutzers mittels WLAN haben wir �ber eine f�r den 
Nutzer passive Positionierung recherchiert. Alle anderen 
Positionierungsmethoden ben�tigten eine Eingabe des Nutzers. Wir haben hier 
die Eingabe der Position auf einer Karte und das Einlesen von QR-Codes oder 
NFC-Tags behandelt. Die Positionierung erm�glicht es allerdings im Hintergrund 
zu laufen und ohne Eingabe des Nutzers die Position zu bestimmen.\\
WLAN ist zur Positionierung innerhalb von Geb�uden geeignet, da es zum einen 
eine weit verbreitete Infrastruktur ist, auf vielen mobilen Plattformen 
verf�gbar ist, W�nde durchdringt und Standard WLAN Access Points bereits eine 
Lokalisierung auf Raum-Genauigkeit erm�glicht.\\
Aus all diesen Gr�nden haben wir uns mit der Positionierung mittels WLAN 
besch�ftigt. 

\subsubsection{Sammeln von WLAN Fingerprints}
Um sp�ter Vergleiche im Client anstellen zu k�nnen mussten wir zuerst Daten 
des Netzwerkes sammeln. Ein Access Point wird dabei eindeutig durch eine 
\textbf{BSSID} gegenzeichnet und der Client gibt Auskunft �ber die empfangene 
Signalst�rke (\textbf{RSS \footnote{RSS: received signal strength wird in dBm 
gemessen.}}), welche beobachtet und aufgezeichnet werden kann.\\
Ziel dieser Phase war es an m�glichsten vielen Punkten in der Hochschule die 
\textit{RSS} zu messen und diese zu einem Punkt auf der Karte der Hochschule 
zu speichern.\\
\missingfigure{}

\subsubsection{Kalibrierung}
Da das Sammeln der WLAN Fingerprints mit einem Smartphone realisiert wird und 
wir davon ausgehen mussten, dass nicht jeder Nutzer das gleiche Smartphone 
besitzt, mussten wir uns eine M�glichkeit der Kalibrierung �berlegen. Dazu 
haben wir �berlegt, dass der Nutzer zuerst in einer Kalibrierungsphase selbst 
einen Fingerprint erstellt von einem von uns festgelegten Ort und diesen mit 
dem von uns gemessenen Fingerprint vergleicht. Dadurch bekommen wir einen 
Faktor um den das Smartphone des Nutzer von unserem Ger�t abweicht. Da wir 
vermuten, dass die WLAN Antennen der Smartphones auch in verschiedenen 
Bereichen, hohe, mittlere und niedrige Signalst�rke, sich stark unterscheiden 
berechnen wir diesen Faktor f�r die gerade genannten Bereiche.\\
Dieser Teilbereich ist in der \textit{StudMap-App} umgesetzt.

\subsubsection{Positionierung mittels WLAN Fingerprints}
Um die Position eines Nutzers ermitteln zu k�nnen muss dieser, wie der 
\textit{Collector} einen Fingerprint des WLANs an seiner aktuellen Position 
erstellen. Diesen Fingerprint und seine Faktoren, welche w�hrend der 
Kalibrierung ermittelt wurden, schickt der Client zum Server, welcher durch 
Vergleiche den Standpunkt ermittelt und zur�ckgibt.\todo{Wie werden die FP 
verglichen.}\\
Dieses Feature ist auch in der \textit{StudMap-App} umgesetzt.