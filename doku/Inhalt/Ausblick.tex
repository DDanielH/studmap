%Erweiterung der Navigation.. flie�ende Navigation mit Richtungsangaben, Distanzangaben, Sprachausgabe
%Verkn�pfung mit weiteren Informationen wie Stundenpl�nen, Sprechstundenzeiten usw.
%Ausweitung auf andere Locations
%Positionsermittlung von Freunden und Navigation zu diesen
%Anbindung von Social Networks
%Sticky Notes in virtuellen R�umen


\chapter {Ausblick}
Nach Abschluss unserer Projektarbeit bleiben noch viele innovative Ideen offen. Dazu geh�ren Dinge, die in handels�blichen Navigationsger�ten zum Standard geh�ren genauso wie kreative Ideen, die das heutige Internetzeitalter erm�glicht.

Entsprechend der Erwartung eines Benutzers, sollte eine Navigation flie�end stattfinden, das hei�t anhand der aktuellen Position werden richtungsweisende Pfeile eingeblendet, eine Sprachausgabe unterst�tzt die Wegfindung und eine Angabe �ber die noch zu gehende Distanz informiert den Benutzer. Diese und �hnliche Funktionen kennt man bereits aus handels�blichen Navigationsger�ten und k�nnte sich eben bei diesen Denkanst��e f�r Erweiterungen einholen.

Grunds�tzlich besteht die M�glichkeit unser System auch in anderen Geb�uden wie Universit�ten oder �ffentlichen Einrichtungen einzusetzen. Es erfordert lediglich eine entsprechende Administration und Einrichtung des Systems.

Des Weiteren bietet es sich an, die in einem Knoten gespeicherten Informationen deutlich zu erweitern. Da g�be es zum einen die Idee, Stundenpl�ne oder Sprechstundenzeiten mit den Knoten zu verkn�pfen oder zumindest auf externe Anwendungen, die den gew�nschten Zweck erf�llen, zu verlinken. Zum anderen k�nnte man jedem Knoten eine virtuelle Pinnwand zuweisen, auf denen jegliche Art von �ffentlichen oder auch privaten Nachrichten hinterlassen werden k�nnen. Private Nachrichten lie�en sich erm�glichen, wenn man eine Art Freundesliste pflegt. Die bereits registrierten Mitglieder bieten dazu eine gute Grundlage.
Dar�ber hinaus w�rde eine solche Freundesliste den Weg ebnen, den Aufenthaltsort von Freunden in dem Geb�ude sehen und sich dorthin navigieren lassen zu k�nnen.

Die bereits vorgestellten sind selbstredend nicht alle denkbaren Erweiterungsm�glichkeiten.
Der Fantasie und dem Erfindergeist der Entwickler sind keine Grenzen
gesetzt mit Ausnahme derer, dass die Anwendung benutzbar und benutzerfreundlich
bleiben sollte.

Um eine letzte Vision mit auf den Weg zu geben, w�re dem Benutzer vielleicht daran
gelegen Informationen in sozialen Netzwerken teilen zu k�nnen. Eine M�glichkeit dies mittels eines einfachen Knopfdruckes zu erledigen, ist eine von vielen verbleibenden Komfortm�glichkeiten.