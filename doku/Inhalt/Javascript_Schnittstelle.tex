\chapter{Javascript Schnittstelle}
\label{cha:jsSchnittstelle}

Die Javascript Schnittstelle der Karte bietet mehrere Funktionen zur Steuerung. Darunter Methoden zur Ver�nderung von Punkten, Setzen von Start- und Endpunkten, Zoomen zu Punkten und das Zur�cksetzen der Karte.

Im folgenden werden diese hier beschrieben.

\section{setStartPoint(nodeId)}
\label{object:setStartPoint}
Setzt einen Startpunkt auf die Karte und markiert diesen in blau.
Sind Start-und-Endpunkt gesetzt, wird der Weg angezeigt.
Eigenschaften:\\
\begin{tabularx}{\textwidth}{|l|X|}
\hline nodeId & ID des Knotens der Kante. \\ 
\hline
\end{tabularx} 

\section{setEndPoint(nodeId)}
\label{object:setEndPoint}
Setzt einen Endpunkt auf die Karte und markiert diesen in rot.
Sind Start-und-Endpunkt gesetzt, wird der Weg angezeigt.

Eigenschaften:\\
\begin{tabularx}{\textwidth}{|l|X|}
\hline nodeId & ID des Knotens der Kante. \\ 
\hline
\end{tabularx} 

\section{highlightPoint(nodeId, radius)}
\label{object:highlightPoint}
Highlighted den �bergebenen Knoten und ver�ndert den Radius.

Eigenschaften:\\
\begin{tabularx}{\textwidth}{|l|X|}
\hline nodeId & ID des Knotens der Kante. \\
\hline radius & Radius den der Knoten haben soll. \\ 
\hline
\end{tabularx} 


\section{clearMap()}
\label{object:clearMap}
Setzt alle Elemente auf der Karte zur�ck.


\section{resetMap()}
\label{object:resetMap}
Setzt alle Elemente auf der Karte und die Start-und-Endpunkte zur�ck.


\section{resetZoom()}
\label{object:resetZoom}
Zoomt die Karte auf Bildschirmf�llende Gr��e.

\section{zoomToNode(nodeId)}
\label{object:zoomToNode}
Zoomt zum �bergebenen Knoten.

Eigenschaften:\\
\begin{tabularx}{\textwidth}{|l|X|}
\hline nodeId & ID des Knotens der Kante. \\
\hline
\end{tabularx} 

