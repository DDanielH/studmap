\chapter{Server}
Der Webserver ist unter 193.175.199.115 erreichbar:

\begin{itemize}
\item StudMap.Admin: \href{http://193.175.199.115/StudMapAdmin}{http://193.175.199.115/StudMapAdmin}
\item StudMap.Client: \href{http://193.175.199.115/StudMapClient}{http://193.175.199.115/StudMapClient}
\item StudMap.Service: \href{http://193.175.199.115/StudMapService}{http://193.175.199.115/StudMapService}
\end{itemize}


\section{Software}
Auf dem Server sind folgende Softwarekomponenten installiert:
\begin{itemize}
\item Internet Information Services (Version 7.5)
\item Microsoft SQL Server 2012
\end{itemize}

\section{Einrichtung IIS}
Der IIS kann sowohl auf einem Windows Server als auch auf einer normalen 
Windows Installation eingerichtet werden. Dazu muss unter "`Programme"'
"`Windows-Features installieren oder deinstallieren"' ausgew�hlt werden. In 
dem Dialog kann der IIS  installiert werden. Da wir in unserem Projekt die 
ASP.NET Technologie verwenden muss diese am IIS installiert werden.\\
Nach der Installation ist der IIS-Manager installiert worin Webseiten 
eingerichtet werden k�nnen.\\
Da wir auf dem Server eine normale Windows Installation laufen haben, kann 
kein Web Deploy aufgespielt werden. Daher m�ssen in Visual Studio Deployment 
Packages erstellt werden, welche �ber einen Rechtsklick auf 
"`StudMap"'>"'Bereitstellen"'>"'Anwendung importieren"' importiert werden 
k�nnen.

\section{Einrichtung SQL Server}
Im Projekt wird ein Microsoft SQL Server 2012 verwendet. Da unsere Anwendung 
auf komplexe Managementfunktionen verzichtet kann auch die kostenlose Express 
Version genutzt werden.

Der Aufbau der Datenbank kann aus den unten aufgef�hrten Datenbankskripten 
hergestellt werden.

\lstinputlisting[language=SQL]{Inhalt/Datenbank/script.sql}

