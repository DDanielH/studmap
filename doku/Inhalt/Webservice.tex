\chapter{Webservice}
F�r den Zugriff auf die Daten stellen wir den Webservice \texttt{StudMap.Service} zur Verf�gung.
Der Webservice besteht dabei aus zwei Schnittstellen in Form von sogenannten Controller Klassen, die jeweils von der Klasse \texttt{ApiController} \footnote{siehe: \href{http://msdn.microsoft.com/en-us/library/system.web.http.apicontroller(v=vs.108).aspx}{MSDN Dokumentation}}  abgeleitet sind:
\begin{itemize}
\item \texttt{MapsController}: Verwaltung von Karten- und Routeninformationen
\item \texttt{UsersController}: Verwaltung von Benutzerinformationen
\end{itemize}

Bevor nun die Funktionen der jeweiligen Controller Klasse erl�utert werden, folgt eine �bersicht, �ber die verschiedenen R�ckgabe Werte und ihre Bedeutung.

\section{R�ckgabewerte}

\subsection{BaseResponse}
\label{BaseResponse}
Allgemeine R�ckgabe vom Service, die einen Status und ggf. einen Fehler enth�lt. Die Daten werden im JSON Format zur�ck gegeben.

\textbf{Beispiel:}
\begin{verbatim}
{ "Status":1, "ErrorCode":0 }
\end{verbatim}

\subsubsection{ResponseStatus}
Wir unterscheiden zwischen:
\begin{itemize}
\item \texttt{None = 0}: Defaultwert
\item \texttt{Ok = 1}: Funktion erfolgreich ausgef�hrt
\item \texttt{Error = 2} Fehler bei Funktionsausf�hrung
\end{itemize}

\subsubsection{ResponseError}
Ist beim \texttt{ResponseStatus Error} gesetzt. Es werden folgende Fehlerszenarien unterschieden:

\textbf{Allgemein:}
\begin{itemize}
\item \texttt{001 - DatabaseError}:\\
Fehler bei der Ausf�hrung einer Datenbankabfrage.
\end{itemize}

\textbf{Registrierung:}
\begin{itemize}
\item \texttt{101 - UserNameDuplicate}:\\
Der Benutzername ist bereits vergeben.
\item \texttt{102 - UserNameInvalid}:\\
Der Benutzername ist ung�ltig.
\item \texttt{103 - PasswordInvalid}:\\
Das Passwort ist ung�ltig.
\end{itemize}

\textbf{Anmeldung:}
\begin{itemize}
\item \texttt{110 - LoginInvalid}:\\
Die Logindaten (Name oder Passwort) sind ung�ltig.
\end{itemize}

\textbf{Maps:}
\begin{itemize}
\item \texttt{201 - MapIdDoesNotExist}:\\
Zur angeforderten MapId existiert keine Map.
\item \texttt{202 - FloorIdDoesNotExist}:\\
Zur angeforderten FloorId existiert kein Floor.
\end{itemize}

\subsection{ObjectResponse}
\label{ObjectResponse}
Eine generische Klasse die \nameref{BaseResponse} um ein Feld \texttt{Object} erweitert, indem die Nutzdaten gespeichert werden.

Beispiel:
\todo{ObjectResponse als JSON String einf�gen}

\subsection{ListResponse}
\label{ListResponse}
Eine generische Klasse die \nameref{BaseResponse} um eine Liste \texttt{List} erweitert, indem Nutzdaten in Form einer Collection gespeichert werden.

Beispiel:
\begin{verbatim}
{"List":[],"Status":1,"ErrorCode":0}
\end{verbatim}

\section{MapsController}
\todo{MapsController dokumentieren}

\section{UsersController}
\label{Webservice_UsersController}
Der \texttt{UsersController} stellt klassische Funktionen zur Benutzerverwaltung zur Verf�gung.

\subsection{Register}
Registriert einen neuen Anwender in der Benutzerrolle \textit{Benutzer}.

Aufruf �ber HTTP Post:\\
\url{http://localhost:1129/api/Users/Register?userName=test&password=geheim}

�bergabe Parameter:\\
userName: Der Benutzername.\\
password: Passwort im Klartext.

R�ckgabewert: \nameref{BaseResponse}

\subsection{Login}
Meldet einen bereits registrierten Anwender am System an.

Aufruf �ber HTTP Post:\\
\url{http://localhost:1129/api/Users/Login?userName=test&password=geheim}

�bergabe Parameter:\\
userName: Der Benutzername.\\
password: Passwort im Klartext.

R�ckgabewert: \nameref{BaseResponse}

\subsection{Logout}
Meldet einen angemeldeten Anwender vom System ab.

Aufruf �ber HTTP Get:\\
\url{http://localhost:1129/api/Users/Logout?userName=test}

�bergabe Parameter:\\
userName: Der Benutzername.\\

R�ckgabewert: \nameref{BaseResponse}

\subsection{GetActiveUsers}
Ermittelt eine Liste der aktuell am System angemeldeten Anwender.

Aufruf �ber HTTP Get:\\
\url{http://localhost:1129/api/Users/GetActiveUsers}

R�ckgabewert: \nameref{ListResponse}