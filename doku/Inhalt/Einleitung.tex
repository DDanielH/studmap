\chapter{Einleitung}
Im Fach Fortgeschrittene Internetanwendungen haben wir uns im Rahmen einer studentischen Projektarbeit mit dem Thema Navigation und Lokalisierung innerhalb von Geb�uden besch�ftigt. Dabei beschr�nkte sich das Ziel unseres Projekts auf die Navigation im Geb�ude der Westf�lischen Hochschule am Campus Bocholt.
Dazu stellten wir uns zu Projektbeginn eine Karte des Geb�udes vor, auf der alle m�glichen Navigationsziele eingezeichnet sind. Bei der Auswahl eines Navigationsziels sollte, nach unseren Vorstellungen, eine entsprechende Wegbeschreibung eingeblendet werden, die uns von unserem aktuellen Standpunkt zum gew�nschten Ziel f�hrt.

Um dieses Ziel zu erreichen, mussten wir uns mit verschiedenen Problemstellungen auseinandersetzen. Zun�chst einmal war es n�tig das Geb�ude vollst�ndig in einem (unserem) System zu erfassen und dieses auf der Karte unserer Vorstellung darzustellen. Zum anderen mussten wir die aktuelle Position (innerhalb des Geb�udes) ermitteln, um diese ebenfalls auf der Karte abbilden zu k�nnen.

\section{Projektorganisation}
Als Plattform f�r unser Projekt verwenden wir Google Code:\\
\href{https://code.google.com/p/studmap/}{https://code.google.com/p/studmap/}

Dort nutzen wir das SVN Repository zur Quellcode Ablage und den Issue Tracker zur Verwaltung von Benutzeranforderungen und Fehlern. Wir haben uns in unserem Projekt f�r eine agile Projektorganisation nach dem Vorbild von Scrum entschieden und den Issue Tracker entsprechend konfiguriert. So stehen uns die Issue Typen User Story, Task und Bug zur Verf�gung. Zus�tzlich haben wir noch vier Kategorien eingef�hrt: ProductBacklog, SprintBacklog, OpenBugs und OpenTasks. Mittels der Kategorien k�nnen wir die verschiedenen Issues besser strukturieren.

Kurz nach Beginn des Projektes haben wir die Benutzeranforderungen in Form von User Stories angelegt und dem ProductBacklog zugewiesen. F�r dieses Projekt haben wir uns auf Sprints mit einer Dauer von jeweils zwei Wochen geeinigt. Zu Beginn eines jeden Sprints haben wir entsprechende User Stories in den SprintBacklog �bertragen und abgearbeitet.