\chapter{Verwendung des Webservices}
\todo{Best Practices "Programmier-Handbuch" f�r MapsController schreiben.}

\section{Verwendung der Benutzerschnittstelle}
\label{Verwendung_Benutzerschnittstelle}
\subsection{Registrierung}
Bevor sich ein Benutzer am StudMap System anmelden kann, muss er sich zun�chst �ber die Funktion \nameref{service:Register} registrieren.

\subsection{Aktive und inaktive Benutzer}
Im StudMap System wird zwischen aktiven und inaktiven Benutzern unterschieden. Nachdem sich ein Benutzer am System registriert hat gilt dieser als inaktiv. �ber die Funktion \nameref{service:Login} kann er sich am System anmelden und gilt somit als aktiv.

Damit der angemeldete Benutzer auch aktiv bleibt, sollte sich dieser in einem Zeitintervall von f�nf Minuten �ber die Methode \nameref{service:Login} am System aktiv melden. Nach einer Inaktivit�t von 15 Minuten wird der Benutzer automatisch inaktiv.

�ber die Funktion \nameref{service:Logout} kann sich ein Benutzer wieder vom System abmelden und wird somit inaktiv.

\subsection{Aktive Benutzer abfragen}
Die aktiven Benutzer k�nnen �ber die Funktion \nameref{service:GetActiveUsers} abgefragt werden. Damit die Anzeige der aktiven Benutzer im Client m�glichst aktuell ist, sollte diese Abfrage ebenfalls in regelm��igen Zeitabst�nden erfolgen.