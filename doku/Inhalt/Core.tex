\chapter{Core Bibliothek}
\label{cha:Core-Bib}
Wir haben schon fr�h festgestellt, dass es eine Vielzahl an Strukturen und Funktionalit�ten geben wird, 
die sowohl in der Collector-Applikation als auch in unserer Navigator-Applikation f�r den Endbenutzer Anwendung finden.
Dem entsprechend haben wir diese in eine eigene Android-Bibliothek ausgelagert, die wir wiederum in unsere Android-Applikationen einbinden konnten.
Zu den Kernfunktionalit�ten und -Strukturen geh�ren unter Anderem folgende:
\begin{itemize}
\item Grundlegende Definitionen einer Map, eines Floors oder eines Knoten
\item Konstanten f�r die Kommunikation mit dem Webservice
\item Ein Errorhandler f�r alle grundlegenden Fehler
\item Snippets zur einfachen Kommunikation mit dem Benutzer mittels Dialogen o.�.
\item Abbildung des Webservices zur vereinfachten Nutzung
\item Javascript-Schnittstellendefinitionen f�r die Interaktion auf der Karte
\item Asynchrone Tasks f�r Webservicekommunikation inkl. entsprechender Listener
\end{itemize}

\section{Map Webview}
\label{cha:MapWebview}
Zur Anzeige der Karte nutzen wie die Bibliothek d3 und Floormap. Dadurch ist es m�glich in einem Bild, mithilfe von Touch-Gesten, hinein und heraus zu zoomen. D3 nutzt dazu ein SVG Image welches auch erweiterbar ist. Wir haben dazu noch Funktionen erg�nzt die es uns erm�glicht einen Graphen mit Knoten und Kanten anzuzeigen. Zur anzeige nutzen wir Kreis und Linien Elemente von SVG. Zus�tzlich ist es m�glich Punkte zu ver�ndern um diese deutlicher zu machen, einen Start und Endpunkt zur Navigation fest zu legen, zu einem Punkt zu zoomen und alles zur�ck zu setzen. 

Eine genauere �bersicht �ber die Schnittstellenbeschreibung ist im Anhang: \nameref{cha:jsSchnittstelle} enthalten.