\chapter{Core Bibliothek}
\label{cha:Core-Bib}
Wir haben schon fr�h festgestellt, dass es eine Vielzahl an Strukturen und Funktionalit�ten geben wird, 
die sowohl in der Collector-Applikation als auch in unserer Navigator-Applikation f�r den Endbenutzer Anwendung finden.
Dementsprechend haben wir diese in eine eigene Android-Bibliothek ausgelagert, die wir wiederum in unsere Android-Applikationen einbinden konnten.
Zu den Kernfunktionalit�ten und -Strukturen geh�ren unter anderem folgende:
\begin{itemize}
\item Grundlegende Definitionen einer Karte, eines Stockwerks oder eines Knotens
\item Konstanten f�r die Kommunikation mit dem Webservice
\item Ein Errorhandler f�r alle grundlegenden Fehler
\item Snippets zur einfachen Kommunikation mit dem Benutzer mittels Dialogen
\item Abbildung des Webservices zur vereinfachten Nutzung
\item Javascript-Schnittstellendefinitionen f�r die Interaktion auf der Karte
\item Asynchrone Tasks f�r Webservicekommunikation inkl. entsprechender Listener
\end{itemize}

\section{Map WebView}
\label{cha:MapWebview}
Zur Anzeige der Karte in einer WebView oder einem Browser, nutzen wir die Bibliothek
\textit{D3.js}\footnote{\href{http://d3js.org/}{http://d3js.org/}}
und \textit{Floor Plan}\footnote{\href{http://dciarletta.github.io/d3-floorplan/}{http://dciarletta.github.io/d3-floorplan/}}. Dies erm�glicht uns mit Hilfe von Pitch und Zoom-Gesten eine Touch-Navigation auf der Karte.

\textit{D3} ist eine Javascript Bibliothek zur Anzeige von Graphen und Bildern, die durch das \textit{Floor Plan} Plugin um eine Indoor-Komponente erweitert wurde. Dies realisiert \textit{D3} �ber ein SVG Bild.

Wir haben zus�tzliche Funktionen erg�nzt, die es uns erm�glichen, einen Graphen mit Knoten und Kanten  SVG Kreis- und Linienelemente abzubilden. Au�erdem gibt es die M�glichkeiten, Punkte zu highlighten, einen Start und
Endpunkt zur Navigation zu setzen, zu einem Punkt zu zoomen und die Karte zur�ckzusetzen.

Eine genauere �bersicht �ber die Schnittstellenbeschreibung ist im Anhang \nameref{cha:jsSchnittstelle} enthalten.