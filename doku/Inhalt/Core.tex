\chapter{Core Bibliothek}
\label{cha:Core-Bib}
Wir haben schon fr�h festgestellt, dass es eine Vielzahl an Strukturen und Funktionalit�ten geben wird, 
die sowohl in der Collector-Applikation als auch in unserer Navigator-Applikation f�r den Endbenutzer Anwendung finden.
Dem entsprechend haben wir diese in eine eigene Android-Bibliothek ausgelagert, die wir wiederum in unsere Android-Applikationen einbinden konnten.
Zu den Kernfunktionalit�ten und -Strukturen geh�ren unter Anderem folgende:
\begin{itemize}
\item Grundlegende Definitionen einer Map, eines Floors oder eines Knoten
\item Konstanten f�r die Kommunikation mit dem Webservice
\item Ein Errorhandler f�r alle grundlegenden Fehler
\item Snippets zur einfachen Kommunikation mit dem Benutzer mittels Dialogen
\item Abbildung des Webservices zur vereinfachten Nutzung
\item Javascript-Schnittstellendefinitionen f�r die Interaktion auf der Karte
\item Asynchrone Tasks f�r Webservicekommunikation inkl. entsprechender Listener
\end{itemize}

\section{Map Webview}
\label{cha:MapWebview}
Zur Anzeige der Karte in einem Webview oder einem Browser, nutzen wie die Bibliothek d3 und Floorplan. Dies erm�glicht uns, mithilfe von Pitch und Zoom-Gesten eine Touch Navigation auf der Karte.

D3 ist eine Javascript Bibliothek zur anzeige von Graphen und Bildern, die durch das Floorplan Plugin um eine Indoor Komponente erweitert wurde. Dies realisiert d3 �ber ein SVG Image.

Wir haben noch zus�tzlich Funktionen erg�nzt, die es uns erm�glichen einen Graphen mit Knoten und Kanten  �ber Kreise und Linien SVG Elemente abzubilden. Andere zus�tzliche Funktionalit�ten sind die M�glichkeiten Punkte zu highlighten, einen Start und
Endpunkt zur Navigation zu setzen, zu einem Punkt zu zoomen und die Karte zur�ck
zu setzen

Eine genauere �bersicht �ber die Schnittstellenbeschreibung ist im Anhang: \nameref{cha:jsSchnittstelle} enthalten.