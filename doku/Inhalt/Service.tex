\chapter{Service}
Der Webservice stellt die Logik f�r alle Clients bereit. Dies schlie�t den Navigations-Client,
den Collector-Client und die Admin-Oberfl�che mit ein. Er kapselt in erster Linie die Datenbankzugriffe
und nutzt Caching f�r h�ufige Anfragen. Es werden Funktionen zur Benutzerverwaltung
sowie zur Navigation bereitgestellt.

\section{Allgemeine Struktur}
Als Grundlage verwendet der Webservice ein ASP.NET Web API
Projekt\footnote{\url{http://www.asp.net/web-api}}. Der Datenbankzugriff erfolgt
�ber das Entitiy Framework von Microsoft\footnote{ 
\url{http://msdn.microsoft.com/de-de/library/bb399567(v=vs.110).aspx}}.

In einem Web API Projekt k�nnen die Funktionen des Webservice
auf Basis von selbst definierten Datenstrukturen erstellt werden.
Damit muss sich der Entwickler nicht damit besch�ftigen, wie 
diese Objekte vom Client geliefert werden und in welchem Format
die Antwort gesendet wird. Das Serialisieren erfolgt meist
in den Formaten XML oder JSON. Unser Webservice verwendet im
Standard das JSON-Format, da es leichtgewichtiger ist und
somit besser f�r mobile Clients geeignet ist.

Die Funktionalit�t teilt sich in drei Bereiche auf. Das sind die Karten- und Navigationsinformationen, die Benutzerverwaltung
und das WLAN-Fingerprinting (nur experimentell). Diese Bereiche
werden in drei Controllern abgebildet. Ein Controller ist eine Klasse
die Anfragen gegen den Webservice verarbeitet.

\section{HTTP-Schnittstelle}
Der Webservice bietet eine HTTP-Schnittstelle, die von den Clients
genutzt wird. Diese bietet Funktionen �ber die HTTP-Methoden GET
und POST auf bestimmte URLs an.

Der Client sendet einfache Anfrageparameter �ber die URL-Query-Parameter (GET).
Bei komplexeren Anfragen kann er ein JSON-Dokument in den Body der
Anfrage setzen (POST). Als Antwort erh�lt er ein
JSON-Dokument\footnote{Web API unterst�tzt auch XML, dies wird von unseren Clients allerdings nicht verwendet}.

Eine komplette Referenz der Webservice-Funktionen befindet sich
im Anhang dieser Dokumention (siehe \nameref{cha:Webservice}).

\section{Assembly-Schnittstelle}
Da Anwendungen wie die Admin-Oberfl�che und ein Teil der Client-Oberfl�che
immer auf demselben Server laufen wie der Webservice, ist ein Umweg
�ber die HTTP-Schnittstelle aufwendiger als notwendig. Daher verwenden
diese Anwendungen auch die Assembly-Schnittstelle des Webservices.
Dabei werden die Anfragen direkt an die entsprechenden Controller
gestellt. Das Serialisieren und Deserialisieren der Anfrage- und
Antwortobjekte entf�llt somit. F�r dynmaische Anfragen, z.B. in
Javascript, nutzen diese Anwendungen allerdings weiterhin die
HTTP-Schnittstelle. 

Die Assembly-Schnittstelle bietet die gleiche Funktionalit�t wie
die HTTP-Schnittstelle. Alle Funktionen des Webservice stehen
als .NET Methoden der entsprechenden Controller-Klasse zur Verf�gung.

\section{Caching}
Der Webservice nutzt das Caching der HTTP-Laufzeitumgebung, um 
h�ufige Abfragen (z.B. allgemeine Karten- und Stockwerkinformationen)
und komplexe Berechnungen (z.B. Routen zwischen allen Knoten)
effizienter durchzuf�hren. Die Cache-Objekte sind nur f�r ein
bestimmte Dauer g�ltig (aktuell ein Stunde) und werden invalidiert,
falls sich die Stammdaten �ndern.

\section{Sicherheit}
Aus Sicherheitsgr�nden macht es Sinn die HTTP-Schnittstelle vor
der Ver�ffentlichung des Programms auf HTTPS umzustellen, da
bisher jede Kommunikation unverschl�sselt erfolgt.
