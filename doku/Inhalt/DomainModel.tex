\chapter{Dom�nenmodell}
Durch das Dom�nenmodell legen wir Begriffe fest, mit denen
die Kommunikation im Projektteam einfacher wird.

\section{Anwendungsstruktur}

\subsection{Webservice (StudMap.Service)}
Stellt Funktionen zur Ablage und Abfrage von Navigationsinformationen und Benutzerdaten �ffentlich bereit.

\subsection{Admin-Oberfl�che (StudMap.Admin)}
Weboberfl�che zum Anlegen, Bearbeiten von Navigationsinformationen und Benutzerdaten. Die Weboberfl�che kann nur von der Benutzerrolle Administrator bedient werden.

\subsection{Navigations-Client (StudMap.Navigator)}
App zur Anzeige von Karten und Navigation zwischen Wegpunkten. Wird bedient durch den Anwender.

\subsection{Collector-Client (StudMap.Collector)}
App zur Eingabe von Navigationsinformationen. Wird bedient durch Administratoren.

\section{Benutzerrollen}

\subsection{Anwender (User)}
Muss identifiziert sein und verwendet den Navigations-Client.

\subsection{Administrator}
Verwendet Admin-Oberfl�che und den Collector-Client. Muss registriert sein.

\section{Begriffe}

\subsection{Karte (Map)}
Beschreibt das gesamte Geb�ude mit allen Stockwerken.

\subsection{Stockwerk (Floor)}
2-dimensionale Ansicht mit allen Layern der Ebene.

\subsection{Schicht (Layer)}
Es gibt mehrere Schichten, die jeweils Detailinformationen zu
einem Stockwerk enthalten.

\begin{itemize}
\item Bild-Layer: Enth�lt grafische Darstellung des Stockwerks.
\item Graph-Layer: Enth�lt Kanten und Knoten f�r Routen.
\item POI-Layer: Zusatzinformationen zu speziellen Orten.
\item Routen-Layer: Darstellung grafischer Elemente zur Navigation.
\item Personen-Layer: Darstellung anderer Anwender.
\end{itemize}

\subsection{Route}
Hat einen Start- und einen Endknoten. Verbindet diese beiden Knoten �ber Zwischenknoten und Kanten.

\subsection{Graph}
Gesamtheit aller Knoten und Kanten der Karte (Stockwerk-�bergreifend).

\subsection{Knoten (Node)}
Besteht aus eindeutigem Identifier, X- und Y-Koordinate und Stockwerk. Zu dem Knoten k�nnen zus�tzliche Informationen hinterlegt werden: Name, Raumnummer, NFC-Tag, QR-Tag, ..., Verweis auf POI.

\subsection{Kante (Edge)}
Verbindung zweier Knoten. Bedeutet, dass man von einem Punkt zum anderen laufen kann.

\subsection{Point of Interest (POI)}
Ort besonderen Interesses (z.B. Bibliothek, Mensa, ...)
